\begin{filecontents*}{inputs/refs.bib}
@book {MR1275826,
    AUTHOR = {Crole, Roy L.},
     TITLE = {Categories for types},
    SERIES = {Cambridge Mathematical Textbooks},
 PUBLISHER = {Cambridge University Press, Cambridge},
      YEAR = {1993},
     PAGES = {xviii+335},
      ISBN = {0-521-45092-6; 0-521-45701-7},
   MRCLASS = {18D99 (03B15 03B40 03B70 03G30 18C10 68Q55)},
  MRNUMBER = {1275826 (95h:18007)},
MRREVIEWER = {Ji{\v{r}}{\'{\i}} Ad{\'a}mek},
}
@book {MR3012378,
  AUTHOR = {Dowek, Gilles},
  TITLE = {Proofs and algorithms},
  SERIES = {Undergraduate Topics in Computer Science},
  NOTE = {An introduction to logic and computability,
    Translated from the French by Maribel Fernandez},
  PUBLISHER = {Springer, London},
  YEAR = {2011},
  PAGES = {xii+155},
  ISBN = {978-0-85729-120-2; 978-0-85729-121-9},
  MRCLASS = {03-01 (68-01)},
  MRNUMBER = {3012378},
  DOI = {10.1007/978-0-85729-121-9},
  URL = {http://dx.doi.org/10.1007/978-0-85729-121-9},
}
\end{filecontents*}
\documentclass[11pt]{amsart}
% The following \documentclass options may be useful:
% preprint      Remove this option only once the paper is in final form.
% 10pt          To set in 10-point type instead of 9-point.
% 11pt          To set in 11-point type instead of 9-point.
% numbers       To obtain numeric citation style instead of author/year.

%% \usepackage{setspace}\onehalfspacing

\usepackage{amsmath}
\usepackage{amscd,amssymb,amsthm} %, amsmath are included by default
\usepackage{latexsym,stmaryrd,mathrsfs,enumerate,scalefnt,ifthen}
\usepackage{mathtools}
\usepackage[mathcal]{euscript}
\usepackage[colorlinks=true,urlcolor=black,linkcolor=black,citecolor=black]{hyperref}
\usepackage{url}
\usepackage{scalefnt}
\usepackage{tikz}
\usepackage{color}
\usepackage[margin=1in]{geometry}
\usepackage{scrextend}

%%////////////////////////////////////////////////////////////////////////////////
%% Theorem styles
\numberwithin{equation}{section}
\theoremstyle{plain}
\newtheorem{theorem}{Theorem}[section]
\newtheorem{lemma}[theorem]{Lemma}
\newtheorem{proposition}[theorem]{Proposition}
\newtheorem{prop}[theorem]{Proposition}
\theoremstyle{definition}
\newtheorem{claim}[theorem]{Claim}
\newtheorem{corollary}[theorem]{Corollary}
\newtheorem{definition}[theorem]{Definition}
\newtheorem{notation}[theorem]{Notation}
\newtheorem{Fact}[theorem]{Fact}
\newtheorem*{fact}{Fact}
\newtheorem{example}[theorem]{Example}
\newtheorem{examples}[theorem]{Examples}
\newtheorem{exercise}{Exercise}
\newtheorem*{lem}{Lemma}
\newtheorem*{cor}{Corollary}
\newtheorem*{remark}{Remark}
\newtheorem*{remarks}{Remarks}
\newtheorem*{obs}{Observation}


%%%%%%%%%%%%%%%%%%%%%%%%%%%%%%%%%%%%%%%%
% Acronyms
%%%%%%%%%%%%%%%%%%%%%%%%%%%%%%%%%%%%%%%%
%% \usepackage[acronym, shortcuts]{glossaries}
%\usepackage[smaller]{acro}
\usepackage[smaller]{acronym}
\usepackage{xspace}

%% \acs{CSP} -- short version of the acronym\\
%% \acl{CSP} -- expanded acronym without mentioning the acronym.\\
%% \acp{CSP} -- plurals.\\
%% \acfp{CSP} -- long forms into plurals.\\
%% \acsp{CSP} -- short form into a plural.\\
%% \aclp{CSP} -- long form into a plural.\\
%% \acfi{CSP} -- Full Name acronym in italics and abbreviated form in upshape.\\
%% \acsu{CSP} -- short form of the acronym and marks it as used.\\
%% \aclu{CSP} -- Prints the long form of the acronym and marks it as used.\\

\acrodef{lics}[LICS]{Logic in Computer Science}
\acrodef{sat}[SAT]{satisfiability}
\acrodef{nae}[NAE]{not-all-equal}
\acrodef{ctb}[CTB]{cube term blocker}
\acrodef{tct}[TCT]{tame congruence theory}
\acrodef{wnu}[WNU]{weak near-unanimity}
\acrodef{CSP}[CSP]{constraint satisfaction problem}
\acrodef{MAS}[MAS]{minimal absorbing subuniverse}
\acrodef{MA}[MA]{minimal absorbing}
\acrodef{cib}[CIB]{commutative idempotent binar}
\acrodef{sd}[SD]{semidistributive}
\acrodef{NP}[NP]{nondeterministic polynomial time}
\acrodef{P}[P]{polynomial time}
\acrodef{PeqNP}[P $ = $ NP]{P is NP}
\acrodef{PneqNP}[P $ \neq $ NP]{P is not NP}

%%% This enables markdown style bold and italic. %%%
\makeatletter
\def\starparse{\@ifnextchar*{\bfstarx}{\itstar}}
\def\bfstarx#1{\@ifnextchar*{\bfitstar\@gobble}{\bfstar}}
\makeatother
\def\itstar#1*{\textit{#1}\starON}
\def\bfstar#1**{\textbf{#1}\starON}
\def\bfitstar#1***{\textbf{\textit{#1}}\starON}
\def\starON{\catcode`\*=\active}
\def\starOFF{\catcode`\*=12}
\starON
\def*{\starOFF \starparse}
\starOFF

%%%%%%%%%%%%%%%%%%%%%%%%%%%%%%%%%%%%%%%%%%%%%%%%%%%%%%%%%%%%%%%%%

\usepackage{inputs/proof-dashed}


%%%%%%%%%%%%%%%%%%%%%%%%%%%%%%%%%%%%%%%%%%%%%%%%%%%%%%%%%%%%%%%%%

%% Put new macros in the macros.sty file
\usepackage{inputs/macros}

% \usepackage[backend=bibtex]{biblatex}
% \bibliography{inputs/refs.bib}

\begin{document}
\starON
\title[Notes on Dowek]{Notes on Dowek's\\
 ``Proofs and Algorithms''}
\date{\today}
\author[W.~DeMeo and H.~Shin]{William DeMeo and Hyeyoung Shin}
\address{University of Hawaii}
\email{williamdemeo@gmail.com}
\email{hyeyoungshinw@gmail.com}

%% \thanks{The authors would like to extend special thanks to...}
\begin{abstract}
This document contains notes on the book~\cite{MR3012378}, {\it Proofs and
Algorithms}, by by Gilles Dowek.  We exerpt the main definitions and theorems,
we attempt to solve some exercises, and we resolve problems we found with some
of the definitions and theorems in the original text. 
\end{abstract}

\maketitle


\part{Proofs}

\section{1 Predicate Logic}

\subsection{Inductive Definitions}

\subsubsection{The Fixed Point Theorem}\

\noindent **Definition 1.2** (Weakly complete ordering) An ordering relation $\leq$ is said to be
weakly complete if each increasing sequence has a limit.\\
\\
**Definition 1.3** (Increasing function) Let $\leq$ be an ordering relation over
a set $E$ and $f$ a function from $E$ to $E$. The function $f$ is increasing if
$x \leq y \Rightarrow f x \leq fy$. \\
\\
**Definition 1.4** (Continuous function) Let $\leq$ be a weakly complete ordering rela-
tion over the set $E$, and $f$ an increasing function from $E$ to $E$. The function $f$ is
continuous if for any increasing sequence $\lim_i (f u_i ) = f (\lim_i u_i )$.\\
\\
**Proposition 1.1** (First fixed point theorem) Let $\leq$ be a weakly complete ordering
relation over a set $E$ that has a least element $m$. Let $f$ be a function from 
$E$ to $E$. If $f$ is continuous then $p = \lim_i (f^i m)$ is the least fixed
point of $f$. \\
\\
**Definition 1.5** (Strongly complete ordering) An ordering relation $\leq$ over a set $E$
is strongly complete if $E$ has a least element 0 and every nonempty subset $A$ of
$E$ has a least upper bound, denoted by $\sup A$.\footnote{Dowek's definition
  does not require that $E$ has a least element and also does not require that
  $A$ be nonempty.  These omissions cause problems.} \\
\\
**Exercise 1.1** Show that any strongly complete ordering is also weakly complete.  
(done)\\
\\
Is the ordering shown below weakly complete? Is it strongly complete?  

\begin{verbatim}
    b       c
     \     /
      \   /
       \ /
        a
\end{verbatim}
(Answers: yes; no.)\\
\\
**Proposition 1.2** If the ordering $\leq$ over the set $E$ is strongly
complete, then any nonempty subset $A$ of $E$ has a greatest lower bound, $\inf A$.\\[4pt]
*Proof.* Let $A$ be a nonempty subset of $E$. $A$ has a least upper bound, 
$\sup  A$. Define $B = \{ y \in E \mid (\forall x \in A)\, y \leq x \}$. 
$B$ is a subset of $E$, it is nonempty since $\sup A$ exists. 
Therefore, $B$ has a least upper bound, say, $l$. Note
\begin{itemize}
\item $(\forall y\in B) \, y \leq l$
\item $\bigl((\forall y\in B)\, y \leq l'\bigr)\; \Rightarrow \; l \leq l'$.
\end{itemize}

Then $l$ is th greatest lower bound of $A$. Why?
\begin{itemize}
\item 
 $l$ is a lower bound of $A$ since if $x \in A$, then $x$ is an upper bound of $B$ and
 $l \leq x$.
\item $l$ is the greatest lower bound since if $y$ is a lower bound of $A$, 
$y$ is in $B$ and $y \leq l$.
\end{itemize}

\noindent **Proposition 1.3** (Second fixed point theorem) Let $\leq$ be a strongly
  complete ordering over a set $E$. Let $f$ be a function from $E$ to E. If
  $f$ is increasing then $p = \inf \{c \mid f c \leq c\}$ is the least fixed point
  of $f$.\\
\\
*Proof.* Let $C$ be the set $\{c \mid f c \leq c\}$ and $c$ be an element of $C$. Then
 $p \leq c$ since $p$ is $\inf C$. Since $f$ is increasing $f p \leq f c$. Note 
 $f c \leq c$ because $c$ is an element of $C$. 

\subsubsection{Inductive Definitions}
\subsubsection{Structural Induction}
\subsubsection{Derivations}
\subsubsection{The Reflexive-Transitive Closure of a Relation}


\subsection{Languages}

\subsubsection{Languages Without Variables}
\subsubsection{Variables}
\subsubsection{Many-Sorted Languages}
\subsubsection{Substitution}
\subsubsection{Articulation}


\subsection{The Languages of Predicate Logic}


\subsection{Proofs}


\subsection{Examples of Theories}


\subsection{Variations on the Principle of the Excluded Middle}
\subsubsection{Double Negation}
\subsubsection{Multi-conclusion Sequents}

\appendix

% \section{Appendix}
%% This is the text of the appendix, if you need one.

%\bibliographystyle{amsplain} %% or amsalpha
%% \bibliographystyle{plain-url}
% \printbibliography
\bibliographystyle{alphaurl}
\bibliography{inputs/refs.bib}


\end{document}
